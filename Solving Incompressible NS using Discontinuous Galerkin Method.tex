
% Default to the notebook output style

    


% Inherit from the specified cell style.




    
\documentclass[11pt]{article}

    
    
    \usepackage[T1]{fontenc}
    % Nicer default font (+ math font) than Computer Modern for most use cases
    \usepackage{mathpazo}

    % Basic figure setup, for now with no caption control since it's done
    % automatically by Pandoc (which extracts ![](path) syntax from Markdown).
    \usepackage{graphicx}
    % We will generate all images so they have a width \maxwidth. This means
    % that they will get their normal width if they fit onto the page, but
    % are scaled down if they would overflow the margins.
    \makeatletter
    \def\maxwidth{\ifdim\Gin@nat@width>\linewidth\linewidth
    \else\Gin@nat@width\fi}
    \makeatother
    \let\Oldincludegraphics\includegraphics
    % Set max figure width to be 80% of text width, for now hardcoded.
    \renewcommand{\includegraphics}[1]{\Oldincludegraphics[width=.8\maxwidth]{#1}}
    % Ensure that by default, figures have no caption (until we provide a
    % proper Figure object with a Caption API and a way to capture that
    % in the conversion process - todo).
    \usepackage{caption}
    \DeclareCaptionLabelFormat{nolabel}{}
    \captionsetup{labelformat=nolabel}

    \usepackage{adjustbox} % Used to constrain images to a maximum size 
    \usepackage{xcolor} % Allow colors to be defined
    \usepackage{enumerate} % Needed for markdown enumerations to work
    \usepackage{geometry} % Used to adjust the document margins
    \usepackage{amsmath} % Equations
    \usepackage{amssymb} % Equations
    \usepackage{textcomp} % defines textquotesingle
    % Hack from http://tex.stackexchange.com/a/47451/13684:
    \AtBeginDocument{%
        \def\PYZsq{\textquotesingle}% Upright quotes in Pygmentized code
    }
    \usepackage{upquote} % Upright quotes for verbatim code
    \usepackage{eurosym} % defines \euro
    \usepackage[mathletters]{ucs} % Extended unicode (utf-8) support
    \usepackage[utf8x]{inputenc} % Allow utf-8 characters in the tex document
    \usepackage{fancyvrb} % verbatim replacement that allows latex
    \usepackage{grffile} % extends the file name processing of package graphics 
                         % to support a larger range 
    % The hyperref package gives us a pdf with properly built
    % internal navigation ('pdf bookmarks' for the table of contents,
    % internal cross-reference links, web links for URLs, etc.)
    \usepackage{hyperref}
    \usepackage{longtable} % longtable support required by pandoc >1.10
    \usepackage{booktabs}  % table support for pandoc > 1.12.2
    \usepackage[inline]{enumitem} % IRkernel/repr support (it uses the enumerate* environment)
    \usepackage[normalem]{ulem} % ulem is needed to support strikethroughs (\sout)
                                % normalem makes italics be italics, not underlines
    

    
    
    % Colors for the hyperref package
    \definecolor{urlcolor}{rgb}{0,.145,.698}
    \definecolor{linkcolor}{rgb}{.71,0.21,0.01}
    \definecolor{citecolor}{rgb}{.12,.54,.11}

    % ANSI colors
    \definecolor{ansi-black}{HTML}{3E424D}
    \definecolor{ansi-black-intense}{HTML}{282C36}
    \definecolor{ansi-red}{HTML}{E75C58}
    \definecolor{ansi-red-intense}{HTML}{B22B31}
    \definecolor{ansi-green}{HTML}{00A250}
    \definecolor{ansi-green-intense}{HTML}{007427}
    \definecolor{ansi-yellow}{HTML}{DDB62B}
    \definecolor{ansi-yellow-intense}{HTML}{B27D12}
    \definecolor{ansi-blue}{HTML}{208FFB}
    \definecolor{ansi-blue-intense}{HTML}{0065CA}
    \definecolor{ansi-magenta}{HTML}{D160C4}
    \definecolor{ansi-magenta-intense}{HTML}{A03196}
    \definecolor{ansi-cyan}{HTML}{60C6C8}
    \definecolor{ansi-cyan-intense}{HTML}{258F8F}
    \definecolor{ansi-white}{HTML}{C5C1B4}
    \definecolor{ansi-white-intense}{HTML}{A1A6B2}

    % commands and environments needed by pandoc snippets
    % extracted from the output of `pandoc -s`
    \providecommand{\tightlist}{%
      \setlength{\itemsep}{0pt}\setlength{\parskip}{0pt}}
    \DefineVerbatimEnvironment{Highlighting}{Verbatim}{commandchars=\\\{\}}
    % Add ',fontsize=\small' for more characters per line
    \newenvironment{Shaded}{}{}
    \newcommand{\KeywordTok}[1]{\textcolor[rgb]{0.00,0.44,0.13}{\textbf{{#1}}}}
    \newcommand{\DataTypeTok}[1]{\textcolor[rgb]{0.56,0.13,0.00}{{#1}}}
    \newcommand{\DecValTok}[1]{\textcolor[rgb]{0.25,0.63,0.44}{{#1}}}
    \newcommand{\BaseNTok}[1]{\textcolor[rgb]{0.25,0.63,0.44}{{#1}}}
    \newcommand{\FloatTok}[1]{\textcolor[rgb]{0.25,0.63,0.44}{{#1}}}
    \newcommand{\CharTok}[1]{\textcolor[rgb]{0.25,0.44,0.63}{{#1}}}
    \newcommand{\StringTok}[1]{\textcolor[rgb]{0.25,0.44,0.63}{{#1}}}
    \newcommand{\CommentTok}[1]{\textcolor[rgb]{0.38,0.63,0.69}{\textit{{#1}}}}
    \newcommand{\OtherTok}[1]{\textcolor[rgb]{0.00,0.44,0.13}{{#1}}}
    \newcommand{\AlertTok}[1]{\textcolor[rgb]{1.00,0.00,0.00}{\textbf{{#1}}}}
    \newcommand{\FunctionTok}[1]{\textcolor[rgb]{0.02,0.16,0.49}{{#1}}}
    \newcommand{\RegionMarkerTok}[1]{{#1}}
    \newcommand{\ErrorTok}[1]{\textcolor[rgb]{1.00,0.00,0.00}{\textbf{{#1}}}}
    \newcommand{\NormalTok}[1]{{#1}}
    
    % Additional commands for more recent versions of Pandoc
    \newcommand{\ConstantTok}[1]{\textcolor[rgb]{0.53,0.00,0.00}{{#1}}}
    \newcommand{\SpecialCharTok}[1]{\textcolor[rgb]{0.25,0.44,0.63}{{#1}}}
    \newcommand{\VerbatimStringTok}[1]{\textcolor[rgb]{0.25,0.44,0.63}{{#1}}}
    \newcommand{\SpecialStringTok}[1]{\textcolor[rgb]{0.73,0.40,0.53}{{#1}}}
    \newcommand{\ImportTok}[1]{{#1}}
    \newcommand{\DocumentationTok}[1]{\textcolor[rgb]{0.73,0.13,0.13}{\textit{{#1}}}}
    \newcommand{\AnnotationTok}[1]{\textcolor[rgb]{0.38,0.63,0.69}{\textbf{\textit{{#1}}}}}
    \newcommand{\CommentVarTok}[1]{\textcolor[rgb]{0.38,0.63,0.69}{\textbf{\textit{{#1}}}}}
    \newcommand{\VariableTok}[1]{\textcolor[rgb]{0.10,0.09,0.49}{{#1}}}
    \newcommand{\ControlFlowTok}[1]{\textcolor[rgb]{0.00,0.44,0.13}{\textbf{{#1}}}}
    \newcommand{\OperatorTok}[1]{\textcolor[rgb]{0.40,0.40,0.40}{{#1}}}
    \newcommand{\BuiltInTok}[1]{{#1}}
    \newcommand{\ExtensionTok}[1]{{#1}}
    \newcommand{\PreprocessorTok}[1]{\textcolor[rgb]{0.74,0.48,0.00}{{#1}}}
    \newcommand{\AttributeTok}[1]{\textcolor[rgb]{0.49,0.56,0.16}{{#1}}}
    \newcommand{\InformationTok}[1]{\textcolor[rgb]{0.38,0.63,0.69}{\textbf{\textit{{#1}}}}}
    \newcommand{\WarningTok}[1]{\textcolor[rgb]{0.38,0.63,0.69}{\textbf{\textit{{#1}}}}}
    
    
    % Define a nice break command that doesn't care if a line doesn't already
    % exist.
    \def\br{\hspace*{\fill} \\* }
    % Math Jax compatability definitions
    \def\gt{>}
    \def\lt{<}
    % Document parameters
    \title{Solving Incompressible NS using Discontinuous Galerkin Method}
    
    
    

    % Pygments definitions
    
\makeatletter
\def\PY@reset{\let\PY@it=\relax \let\PY@bf=\relax%
    \let\PY@ul=\relax \let\PY@tc=\relax%
    \let\PY@bc=\relax \let\PY@ff=\relax}
\def\PY@tok#1{\csname PY@tok@#1\endcsname}
\def\PY@toks#1+{\ifx\relax#1\empty\else%
    \PY@tok{#1}\expandafter\PY@toks\fi}
\def\PY@do#1{\PY@bc{\PY@tc{\PY@ul{%
    \PY@it{\PY@bf{\PY@ff{#1}}}}}}}
\def\PY#1#2{\PY@reset\PY@toks#1+\relax+\PY@do{#2}}

\expandafter\def\csname PY@tok@mf\endcsname{\def\PY@tc##1{\textcolor[rgb]{0.40,0.40,0.40}{##1}}}
\expandafter\def\csname PY@tok@kn\endcsname{\let\PY@bf=\textbf\def\PY@tc##1{\textcolor[rgb]{0.00,0.50,0.00}{##1}}}
\expandafter\def\csname PY@tok@gd\endcsname{\def\PY@tc##1{\textcolor[rgb]{0.63,0.00,0.00}{##1}}}
\expandafter\def\csname PY@tok@ne\endcsname{\let\PY@bf=\textbf\def\PY@tc##1{\textcolor[rgb]{0.82,0.25,0.23}{##1}}}
\expandafter\def\csname PY@tok@mo\endcsname{\def\PY@tc##1{\textcolor[rgb]{0.40,0.40,0.40}{##1}}}
\expandafter\def\csname PY@tok@sd\endcsname{\let\PY@it=\textit\def\PY@tc##1{\textcolor[rgb]{0.73,0.13,0.13}{##1}}}
\expandafter\def\csname PY@tok@nb\endcsname{\def\PY@tc##1{\textcolor[rgb]{0.00,0.50,0.00}{##1}}}
\expandafter\def\csname PY@tok@err\endcsname{\def\PY@bc##1{\setlength{\fboxsep}{0pt}\fcolorbox[rgb]{1.00,0.00,0.00}{1,1,1}{\strut ##1}}}
\expandafter\def\csname PY@tok@s1\endcsname{\def\PY@tc##1{\textcolor[rgb]{0.73,0.13,0.13}{##1}}}
\expandafter\def\csname PY@tok@gs\endcsname{\let\PY@bf=\textbf}
\expandafter\def\csname PY@tok@nn\endcsname{\let\PY@bf=\textbf\def\PY@tc##1{\textcolor[rgb]{0.00,0.00,1.00}{##1}}}
\expandafter\def\csname PY@tok@c\endcsname{\let\PY@it=\textit\def\PY@tc##1{\textcolor[rgb]{0.25,0.50,0.50}{##1}}}
\expandafter\def\csname PY@tok@k\endcsname{\let\PY@bf=\textbf\def\PY@tc##1{\textcolor[rgb]{0.00,0.50,0.00}{##1}}}
\expandafter\def\csname PY@tok@ge\endcsname{\let\PY@it=\textit}
\expandafter\def\csname PY@tok@kt\endcsname{\def\PY@tc##1{\textcolor[rgb]{0.69,0.00,0.25}{##1}}}
\expandafter\def\csname PY@tok@gi\endcsname{\def\PY@tc##1{\textcolor[rgb]{0.00,0.63,0.00}{##1}}}
\expandafter\def\csname PY@tok@kd\endcsname{\let\PY@bf=\textbf\def\PY@tc##1{\textcolor[rgb]{0.00,0.50,0.00}{##1}}}
\expandafter\def\csname PY@tok@nt\endcsname{\let\PY@bf=\textbf\def\PY@tc##1{\textcolor[rgb]{0.00,0.50,0.00}{##1}}}
\expandafter\def\csname PY@tok@ow\endcsname{\let\PY@bf=\textbf\def\PY@tc##1{\textcolor[rgb]{0.67,0.13,1.00}{##1}}}
\expandafter\def\csname PY@tok@nd\endcsname{\def\PY@tc##1{\textcolor[rgb]{0.67,0.13,1.00}{##1}}}
\expandafter\def\csname PY@tok@cp\endcsname{\def\PY@tc##1{\textcolor[rgb]{0.74,0.48,0.00}{##1}}}
\expandafter\def\csname PY@tok@ss\endcsname{\def\PY@tc##1{\textcolor[rgb]{0.10,0.09,0.49}{##1}}}
\expandafter\def\csname PY@tok@gh\endcsname{\let\PY@bf=\textbf\def\PY@tc##1{\textcolor[rgb]{0.00,0.00,0.50}{##1}}}
\expandafter\def\csname PY@tok@gu\endcsname{\let\PY@bf=\textbf\def\PY@tc##1{\textcolor[rgb]{0.50,0.00,0.50}{##1}}}
\expandafter\def\csname PY@tok@gt\endcsname{\def\PY@tc##1{\textcolor[rgb]{0.00,0.27,0.87}{##1}}}
\expandafter\def\csname PY@tok@w\endcsname{\def\PY@tc##1{\textcolor[rgb]{0.73,0.73,0.73}{##1}}}
\expandafter\def\csname PY@tok@s2\endcsname{\def\PY@tc##1{\textcolor[rgb]{0.73,0.13,0.13}{##1}}}
\expandafter\def\csname PY@tok@sc\endcsname{\def\PY@tc##1{\textcolor[rgb]{0.73,0.13,0.13}{##1}}}
\expandafter\def\csname PY@tok@kr\endcsname{\let\PY@bf=\textbf\def\PY@tc##1{\textcolor[rgb]{0.00,0.50,0.00}{##1}}}
\expandafter\def\csname PY@tok@nf\endcsname{\def\PY@tc##1{\textcolor[rgb]{0.00,0.00,1.00}{##1}}}
\expandafter\def\csname PY@tok@sr\endcsname{\def\PY@tc##1{\textcolor[rgb]{0.73,0.40,0.53}{##1}}}
\expandafter\def\csname PY@tok@go\endcsname{\def\PY@tc##1{\textcolor[rgb]{0.53,0.53,0.53}{##1}}}
\expandafter\def\csname PY@tok@nl\endcsname{\def\PY@tc##1{\textcolor[rgb]{0.63,0.63,0.00}{##1}}}
\expandafter\def\csname PY@tok@m\endcsname{\def\PY@tc##1{\textcolor[rgb]{0.40,0.40,0.40}{##1}}}
\expandafter\def\csname PY@tok@s\endcsname{\def\PY@tc##1{\textcolor[rgb]{0.73,0.13,0.13}{##1}}}
\expandafter\def\csname PY@tok@se\endcsname{\let\PY@bf=\textbf\def\PY@tc##1{\textcolor[rgb]{0.73,0.40,0.13}{##1}}}
\expandafter\def\csname PY@tok@vm\endcsname{\def\PY@tc##1{\textcolor[rgb]{0.10,0.09,0.49}{##1}}}
\expandafter\def\csname PY@tok@vg\endcsname{\def\PY@tc##1{\textcolor[rgb]{0.10,0.09,0.49}{##1}}}
\expandafter\def\csname PY@tok@no\endcsname{\def\PY@tc##1{\textcolor[rgb]{0.53,0.00,0.00}{##1}}}
\expandafter\def\csname PY@tok@vc\endcsname{\def\PY@tc##1{\textcolor[rgb]{0.10,0.09,0.49}{##1}}}
\expandafter\def\csname PY@tok@vi\endcsname{\def\PY@tc##1{\textcolor[rgb]{0.10,0.09,0.49}{##1}}}
\expandafter\def\csname PY@tok@gr\endcsname{\def\PY@tc##1{\textcolor[rgb]{1.00,0.00,0.00}{##1}}}
\expandafter\def\csname PY@tok@mh\endcsname{\def\PY@tc##1{\textcolor[rgb]{0.40,0.40,0.40}{##1}}}
\expandafter\def\csname PY@tok@mb\endcsname{\def\PY@tc##1{\textcolor[rgb]{0.40,0.40,0.40}{##1}}}
\expandafter\def\csname PY@tok@il\endcsname{\def\PY@tc##1{\textcolor[rgb]{0.40,0.40,0.40}{##1}}}
\expandafter\def\csname PY@tok@na\endcsname{\def\PY@tc##1{\textcolor[rgb]{0.49,0.56,0.16}{##1}}}
\expandafter\def\csname PY@tok@sx\endcsname{\def\PY@tc##1{\textcolor[rgb]{0.00,0.50,0.00}{##1}}}
\expandafter\def\csname PY@tok@sa\endcsname{\def\PY@tc##1{\textcolor[rgb]{0.73,0.13,0.13}{##1}}}
\expandafter\def\csname PY@tok@c1\endcsname{\let\PY@it=\textit\def\PY@tc##1{\textcolor[rgb]{0.25,0.50,0.50}{##1}}}
\expandafter\def\csname PY@tok@nv\endcsname{\def\PY@tc##1{\textcolor[rgb]{0.10,0.09,0.49}{##1}}}
\expandafter\def\csname PY@tok@kp\endcsname{\def\PY@tc##1{\textcolor[rgb]{0.00,0.50,0.00}{##1}}}
\expandafter\def\csname PY@tok@o\endcsname{\def\PY@tc##1{\textcolor[rgb]{0.40,0.40,0.40}{##1}}}
\expandafter\def\csname PY@tok@fm\endcsname{\def\PY@tc##1{\textcolor[rgb]{0.00,0.00,1.00}{##1}}}
\expandafter\def\csname PY@tok@sb\endcsname{\def\PY@tc##1{\textcolor[rgb]{0.73,0.13,0.13}{##1}}}
\expandafter\def\csname PY@tok@nc\endcsname{\let\PY@bf=\textbf\def\PY@tc##1{\textcolor[rgb]{0.00,0.00,1.00}{##1}}}
\expandafter\def\csname PY@tok@ch\endcsname{\let\PY@it=\textit\def\PY@tc##1{\textcolor[rgb]{0.25,0.50,0.50}{##1}}}
\expandafter\def\csname PY@tok@gp\endcsname{\let\PY@bf=\textbf\def\PY@tc##1{\textcolor[rgb]{0.00,0.00,0.50}{##1}}}
\expandafter\def\csname PY@tok@kc\endcsname{\let\PY@bf=\textbf\def\PY@tc##1{\textcolor[rgb]{0.00,0.50,0.00}{##1}}}
\expandafter\def\csname PY@tok@ni\endcsname{\let\PY@bf=\textbf\def\PY@tc##1{\textcolor[rgb]{0.60,0.60,0.60}{##1}}}
\expandafter\def\csname PY@tok@dl\endcsname{\def\PY@tc##1{\textcolor[rgb]{0.73,0.13,0.13}{##1}}}
\expandafter\def\csname PY@tok@cpf\endcsname{\let\PY@it=\textit\def\PY@tc##1{\textcolor[rgb]{0.25,0.50,0.50}{##1}}}
\expandafter\def\csname PY@tok@mi\endcsname{\def\PY@tc##1{\textcolor[rgb]{0.40,0.40,0.40}{##1}}}
\expandafter\def\csname PY@tok@cs\endcsname{\let\PY@it=\textit\def\PY@tc##1{\textcolor[rgb]{0.25,0.50,0.50}{##1}}}
\expandafter\def\csname PY@tok@cm\endcsname{\let\PY@it=\textit\def\PY@tc##1{\textcolor[rgb]{0.25,0.50,0.50}{##1}}}
\expandafter\def\csname PY@tok@sh\endcsname{\def\PY@tc##1{\textcolor[rgb]{0.73,0.13,0.13}{##1}}}
\expandafter\def\csname PY@tok@bp\endcsname{\def\PY@tc##1{\textcolor[rgb]{0.00,0.50,0.00}{##1}}}
\expandafter\def\csname PY@tok@si\endcsname{\let\PY@bf=\textbf\def\PY@tc##1{\textcolor[rgb]{0.73,0.40,0.53}{##1}}}

\def\PYZbs{\char`\\}
\def\PYZus{\char`\_}
\def\PYZob{\char`\{}
\def\PYZcb{\char`\}}
\def\PYZca{\char`\^}
\def\PYZam{\char`\&}
\def\PYZlt{\char`\<}
\def\PYZgt{\char`\>}
\def\PYZsh{\char`\#}
\def\PYZpc{\char`\%}
\def\PYZdl{\char`\$}
\def\PYZhy{\char`\-}
\def\PYZsq{\char`\'}
\def\PYZdq{\char`\"}
\def\PYZti{\char`\~}
% for compatibility with earlier versions
\def\PYZat{@}
\def\PYZlb{[}
\def\PYZrb{]}
\makeatother


    % Exact colors from NB
    \definecolor{incolor}{rgb}{0.0, 0.0, 0.5}
    \definecolor{outcolor}{rgb}{0.545, 0.0, 0.0}



    
    % Prevent overflowing lines due to hard-to-break entities
    \sloppy 
    % Setup hyperref package
    \hypersetup{
      breaklinks=true,  % so long urls are correctly broken across lines
      colorlinks=true,
      urlcolor=urlcolor,
      linkcolor=linkcolor,
      citecolor=citecolor,
      }
    % Slightly bigger margins than the latex defaults
    
    \geometry{verbose,tmargin=1in,bmargin=1in,lmargin=1in,rmargin=1in}
    
    

    \begin{document}
    
    
    \maketitle
    
    

    

    \section{Background}\label{background}

\subsection{Lagrange interpolation}\label{lagrange-interpolation}

Langrangian interpolation is used very heavily in Galerkin methods and
it is worthwile to go through the theory. Lagrangian interpolation is
essentially interpolating a function using an \(N\)-degree polynomial
such that it exactly satisfies the function at \(N+1\) points.

\subsubsection{1D Lagrange
interpolation}\label{d-lagrange-interpolation}

Let \(f(x)\) be the function we wish to interpolate. Let \(f^N(X)\) be
its order \(N\) interpolationg polynomial which perfectly satisfies
\(f(x)\) on the points \(x_0, x_1, x_2, ... x_n\)

We define the order \(N\) Langrangian at point \(x_i\) as
\[L^N_i(x) = \prod_{j=0, j\neq i}^{N} \frac{x-x_j}{x_j-x_i}\]

It can easily be shown that \[L^N_i(x_j) = \delta_{i}\]

The interpolating function can then be written as
\[f^N(x) = \sum_{i=0}^{N}L_i^N(x)f(x_i)\] Note: * \(f^n(x_i) = f(x_i)\)
* Order of \(L_i^N(x)\) is \(N\)



    \begin{center}
    \adjustimage{max size={0.9\linewidth}{0.9\paperheight}}{Solving Incompressible NS using Discontinuous Galerkin Method_files/Solving Incompressible NS using Discontinuous Galerkin Method_3_0.png}
    \end{center}
    { \hspace*{\fill} \\}
    
    \paragraph{2D Lagrange interpolation}\label{d-lagrange-interpolation}

2D Lagrange interpolation on a rectangular grid is extremely simple.

Let the interpoliating polynomial to the function \(f(x,y)\) be
\(f^{NM}(x,y)\) which satisfies the function of a grid of points formed
by \(x_0, x_1, ... x_N\) and \(y_0, y_1, ... y_M\)

\[L_{ij}^{NM}(x,y) = L_i^{N}(x_i) L_j^{M}(y_j)\]
\[f^{MN}(x,y) = \sum_{i=0}^{N} \sum_{j=0}^{M} L_{ij}^{NM}(x,y) f(x_i,y_j)\]

    Following are the various basis functions for 4*4 equispaced and lobatto
nodes



    \begin{center}
    \adjustimage{max size={0.9\linewidth}{0.9\paperheight}}{Solving Incompressible NS using Discontinuous Galerkin Method_files/Solving Incompressible NS using Discontinuous Galerkin Method_7_0.png}
    \end{center}
    { \hspace*{\fill} \\}
    
    \begin{center}
    \adjustimage{max size={0.9\linewidth}{0.9\paperheight}}{Solving Incompressible NS using Discontinuous Galerkin Method_files/Solving Incompressible NS using Discontinuous Galerkin Method_7_1.png}
    \end{center}
    { \hspace*{\fill} \\}
    


    \begin{center}
    \adjustimage{max size={0.9\linewidth}{0.9\paperheight}}{Solving Incompressible NS using Discontinuous Galerkin Method_files/Solving Incompressible NS using Discontinuous Galerkin Method_9_0.png}
    \end{center}
    { \hspace*{\fill} \\}
    
    \subsection{Differentiation}\label{differentiation}

\subsubsection{1D}\label{d}

Differentiation an interpolated function is extremely simple.
\[f^{'N}(x) = \sum_{i=0}^{N}L_i^{'N}(x)f(x_i)\]

\subsubsection{2D}\label{d-1}

Derivatives in 2D work the same way as in 1D

\[\frac{\partial f^{MN}(x,y)}{\partial x} = \sum_{i=0}^{N} \sum_{j=0}^{M} L_i^{'N}(x_i) * L_j^{M}(y_j) f(x_i,y_j)\]
\[\frac{\partial f^{MN}(x,y)}{\partial y} = \sum_{i=0}^{N} \sum_{j=0}^{M} L_i^{N}(x_i) * L_j^{'M}(y_j) f(x_i,y_j)\]

    \subsection{Integration}\label{integration}

We integrate our functions using gaussian quadrature technique. The
integral is converted into a summation over a fixed number of points.

\[\int f(x) dx = \sum w(i)f(x_i)\]

The points that are used for the quadrature can be optimised to give
higher accuracy. For instance, choosing the Gauss-Legendre points would
let you integrate perfectly over a polynomial of order \(2N-1\) where
\(N\) is the number of points used in the quadrature. Gauss-Lobatto
points integrate perfectly over a polynomial of order \(2N-3\), but have
the added advantage of having the edge points as nodes.

When using the galerkin method, we will be chosing certain nodes to
interpolate a function. The values of the function at these points will
thus be trivially known. It thus makes sense to choose the interpolating
nodes to be the same as the nodes used for integration. We thus use the
Lobatto points for both, interpolation and integration. We need to
accurately now the values of a function at boundaries because this will
be used heavily when calculating fluxes. Moreover, as seen above, the
basis functions behave well when lobatto points are chosen instead of
equispaced points.

    \section{Introduction to Galerkin
methods}\label{introduction-to-galerkin-methods}

The core idea of galerkin methods is to take the inner product of the
known function space with the governing equation. We thus project the
governing equation on our function space. For simplicity consider a just
one "element". The function space we use will be the space defined the
basis functions \(L^{NM}_{ij}(x,y)\). This essentially means that our
function space is all functions of the for \(f(x)g(y)\) where \(f\) and
\(g\) is an order \(N\) and \(M\) polynomials respectively. Let our
governing equation be \[\frac{\partial q}{\partial t} = f(q)\]

Inner product in the function space is defined as
\[<f,g> = \int_{\Omega} fg d\Omega\]

Taking the inner proguct of the governing equation with the function
space, we get:

\[\int_{\Omega}\Psi \frac{\partial q^N}{\partial t}d\Omega = \int_{\Omega}\Psi f(q) d\Omega\]
where \(q^N\) is the discretization of \(q\) on out function space.
\[q^N = \sum_j \psi_j q_j\]

Now, the above relation has to hold for every basis function \(\psi_i\)
int the function space.

Thus,
\[\int_{\Omega}\psi_i \sum_j \psi_j \frac{\partial q_j^N}{\partial t}d\Omega = \int_{\Omega}\Psi f(q) d\Omega\]

Taking the summation out of the integral and taking
\(\frac{\partial q_j^N}{\partial t}\) out,
\[ \sum_j (\int_{\Omega}\psi_i \psi_j d\Omega) \frac{\partial q_j^N}{\partial t} = \int_{\Omega}\Psi f(q) d\Omega\]

This can be reduced to the following expression
\[M \frac{\partial q_j^N}{\partial t} = RHS\]

where M, the "Mass matrix" is defined as
\[M_{ij} = \int_{\Omega}\psi_i \psi_j d\Omega\]

The integral is calculated using the Quadrature method discussed above.
If the interpolating nodes were used as the integration nodes, it would
result in an inexact integration since if the number of points is \(N\)
and the order of the polynomial is \(2(N-1)\). Higher order lobatto
nodes can be calculated for exact integration. Inexact integration leads
to the mass matrix being diagonal (easy to check). This reduces the cost
of inverting it. However, this computation has to be done just once for
the entire simulation.

The treatment of the RHS depends on what term it is

\subsection{Continuous vs. Discontinuous
methods}\label{continuous-vs.-discontinuous-methods}

If we are to have a multi-cell system, we can go about it two ways:

\subsubsection{Continuous Galerkin}\label{continuous-galerkin}

In CG methods, the common boundary of neighboring cells is shared. This
essentially means that all our stored variables will be continuous
functions. Our function-space will be a continuous function that is a
collection of polynomials in every cell.

The problem with this method is that the final matrix vector problem we
get will be coupled. This is because the common boundary will occur in
the equation of two different cells. Thus, to solve this system, we will
need to construct giant matrices for the entire system. We'll have to
carry out matrix-vector computations with these large matrices every
timestep

\subsubsection{Discontinuous Galerkin}\label{discontinuous-galerkin}

An alternative to that is the DG method. The boundaries of each cell are
independant of the neighbours. This allows us to do our computations for
each cell independently. Information flows between cells because of the
flux term (which will be shown in the later formulation). This term,
which would have gone zero everywhere but the system boundary for CG is
now non zero at the edges of the DG cell. The boundaries of the
neighbors will be used when computing the flux term.

One huge advantage of DG methods is the ability to run in parallel. The
method scales up beautifully with increasing number of computing nodes.
Every node can be assigned a set of elements that it has to bother
about. Internode communication will be limited to exchanging the
boundary values. Every node can then just perform the matrix-vector
computations on its own set of cells

\subsection{Difussion}\label{difussion}

Governing equation:
\(\frac{\partial T}{\partial t} = \alpha \nabla^2 T\)

We break this into two equations: \[\Theta = \nabla T \\
\frac{\partial T}{\partial t} = \alpha \nabla \cdot \Theta\]

We first solve for \(\Theta\)
\[ \theta_x = \frac{\partial T}{\partial x} \\ \theta_y = \frac{\partial T}{\partial y}\]

Proceeding as earlier,

\begin{align}
 M\Theta^N &= \int_{\Omega}\Psi \nabla T d\Omega \\
 M_i\Theta^N &= \int_{\Omega}\psi_i \nabla  (\sum_j \psi_j T_j) d\Omega \\
 M_i\Theta^N &=  \sum_j \int_{\Omega}\nabla(\psi_j \psi_i) T_j d\Omega - \sum_j \int_{\Omega}\nabla( \psi_i )\psi_j T_j d\Omega\\
 M_i\Theta^N &= \sum_j ([\psi_i \psi_j]_{\Gamma} T_j - \int_{\Omega}\psi_j\nabla\psi_i T_j d\Omega)\\
\end{align}

The first term will be evaluated at the boundaries. This can be
rewritten as matrix vector equation:

\[M\Theta^N = FT - DT\]

Where F and T are the flux and derivative matrices respectively. F is 0
for points not corresponding to the boundary.
\[F_{ij} = [\psi_i \psi_j]_{\Gamma}\] \[D_{ij} = \psi_j\nabla\psi_i\]

Solving the second part:

\begin{align}
\frac{\partial T}{\partial t} &= \alpha \nabla \cdot \Theta\\
\int_\Omega \Psi \frac{\partial T}{\partial t} d\Omega &= \alpha \int_\Omega \Psi \nabla \cdot \Theta d\Omega\\
M_i\frac{\partial T_j}{\partial t} &= \sum_j ([\psi_i \psi_j]_{\Gamma} \Theta_j - \int_{\Omega}\psi_j\nabla\psi_i T_j d\Omega)\\
\end{align}


    \subsubsection{Example: Temperature distrigution on a 2D conducting
plate with dirichlet
boundaries}\label{example-temperature-distrigution-on-a-2d-conducting-plate-with-dirichlet-boundaries}

The right wall is maintained at \(T = 1\) and all the other walls are
maintained at \(T = 0\). The thermal conductivity (\(\alpha\)) is set at
0.1. The time step for all of the simulations is 0.1ms and the time
horizon is 2sec. The solution is compared to the 2D steady state
temperature distribution that can be computed analytically. The order in
X and Y is varied and the errors are plotted. The first set of plots are
generated using inexact integration and the second set using exact. You
can see that the difference in errors is fairly small.

Note: This is a spectral method. There is only one cell.


    \begin{center}
    \adjustimage{max size={0.9\linewidth}{0.9\paperheight}}{Solving Incompressible NS using Discontinuous Galerkin Method_files/Solving Incompressible NS using Discontinuous Galerkin Method_15_1.png}
    \end{center}
    { \hspace*{\fill} \\}
    

    \begin{center}
    \adjustimage{max size={0.9\linewidth}{0.9\paperheight}}{Solving Incompressible NS using Discontinuous Galerkin Method_files/Solving Incompressible NS using Discontinuous Galerkin Method_16_0.png}
    \end{center}
    { \hspace*{\fill} \\}
    
    Breaking up the domain into multiple cells is beneficial since it
reduces the size of the matrices/increases the number of sampling
points. However, higher number of cells also means that you'll have to
repeat the matrix vector operations that many times.

If we were to use a zeroeth order scheme, it would essentially be like a
finite volume scheme where every cell has a single value.

We will now run the same exaple but will vary the number of elements too
keeping the total number of nodes constant to observe how the error
varies. We will just be using inexact integration for this one.


    \begin{center}
    \adjustimage{max size={0.9\linewidth}{0.9\paperheight}}{Solving Incompressible NS using Discontinuous Galerkin Method_files/Solving Incompressible NS using Discontinuous Galerkin Method_18_0.png}
    \end{center}
    { \hspace*{\fill} \\}
    
    \subsubsection{Example: Diffusion of a Gaussian Temperature
distribution}\label{example-diffusion-of-a-gaussian-temperature-distribution}

A spectral method works poorly for this problem, perhaps because of very
low sampling provided by the lobatto points near the center. Note that
the lower accuracy near the borders is due to the inaccurate boundary
conditions. The analytical solution is for the boundaries being
infinitely far away from the gaussian Temperature distribution




    \begin{center}
    \adjustimage{max size={0.9\linewidth}{0.9\paperheight}}{Solving Incompressible NS using Discontinuous Galerkin Method_files/Solving Incompressible NS using Discontinuous Galerkin Method_22_0.png}
    \end{center}
    { \hspace*{\fill} \\}
    

    % Add a bibliography block to the postdoc
    
    
    
    \end{document}
